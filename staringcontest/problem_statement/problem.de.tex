\problemname{Wettstarren}

\illustration{.4}{img/Staring_contest.pdf}{}

\noindent
Ein Wettstarren ist ein klassischer Kampf der Unerschütterlichkeit, bei dem zwei Personen einander in die Augen starren und dabei einen Gesichtsausdruck von sicherer Gelassenheit bewahren.
Ziel ist es, den Augenkontakt länger aufrechtzuerhalten als der Gegner.
Der Wettbewerb endet, wenn einer der Teilnehmer die Fassung verliert, indem er wegschaut, lächelt, spricht oder kichert.

Als Trainer des nationalen Starwettbewerbs musst du für das bevorstehende Weltfinale die Unerschütterlichkeit aller $n$ Mitglieder deiner Mannschaft bestimmen.
Der $i$-te Sportler kann den Blickkontakt genau $a_i$ Sekunden lang aufrechterhalten, aber diese Werte sind Dir anfangs unbekannt.
Du kannst zum Beispiel eine Mannschaft mit $n=3$ Mitgliedern haben:

\medskip
\begin{tabular}{lll}
  $i$ & Name & $a_i$\\\hline
  1 & Anna &  431 \\
  2 & Esther & 623 \\
  3 & Tony &  121\\
\end{tabular}

\medskip
Wenn die Athleten $i$ und $j$ gegeneinander antreten, dauert die Konfrontation genau $\min(a_i, a_j)$ Sekunden, wobei der schwächere Teilnehmer die Fassung verliert und beide Teilnehmer innerhalb eines Sekundenbruchteils zu lächeln und zu kichern beginnen.
Wenn zum Beispiel Anna gegen Esther antritt, dauert der Wettbewerb $431$~Sekunden.
Für einen außenstehenden Beobachter ist es unmöglich, den tatsächlichen Gewinner der Konfrontation (in diesem Fall Esther) zu bestimmen, nur die Dauer des Wettkampfs ist messbar.

Dein Ziel ist es, die Werte $a_1,\ldots, a_n$ anhand möglichst weniger Wettkämpfe zu schätzen.
Es ist klar, dass die Stärke des stärksten Athleten nie bestimmt werden kann, also darfst du einen der Werte $a_i$ unterschätzen.

\section*{Interaktion}

Dies ist ein interaktives Problem.
Die Interaktion beginnt damit, dass du eine einzelne Zeile einliest, die die ganze Zahl $n$ enthält.
Du kannst dann Abfragen der Form ``\texttt{?} $i$ $j$'' stellen, so dass $1\leq i\leq n$ und $1\leq j\leq n$ und $i\neq j$.
Die Antwort auf eine Abfrage ist eine einzelne ganze Zahl: der Wert $\min(a_i, a_j)$.
Die Interaktion endet damit, dass du eine einzelne Zeile ausgibst, bestehend aus \texttt{!}, gefolgt von $n$ Schätzungen in Form von ganzen Zahlen $b_1$, $\ldots$, $b_n$, getrennt durch Leerzeichen.
Dies muss die letzte Zeile der Ausgabe sein.

Deine Eingabe ist korrekt, wenn $b_i=a_i$ für jeden Teilnehmer~$i$ außer einem, den du vielleicht unterschätzt, gilt.
Um genau zu sein, verlangen wir $b_i\leq a_i$ für alle $1\leq i\leq n$ und erlauben $b_k \neq a_k$ für höchstens einen~$k$.

Der Interaktor ist \emph{nicht adaptiv}, d.h. die $a_1,\ldots, a_n$ sind bestimmt, bevor die Interaktion beginnt.

\section*{Beschränkungen und Bewertung}

Die Anzahl $n$ der Athleten erfüllt 
$2\leq n\leq 1500$. % constraint:n
Die Unerschütterlichkeit~$a_i$ eines jeden Athleten erfüllt 
$1\leq a_i\leq 86\,400$; % constraint:skillbounds
sie sind alle unterschiedlich. % constraint:allskillsdifferent
Du kannst 
höchstens $3000$~Abfragen machen; % constraint:maxnumqueries
Deine letzte Ausgabezeile, d.h. die Zeile, die mit \texttt{!} beginnt, wird nicht als Abfrage gezählt.

Deine Lösung wird an einer Reihe von Testgruppen getestet, von denen jede eine bestimmte Anzahl von Punkten wert ist.
Jede Testgruppe enthält eine Reihe von Testfällen.
Um die Punkte für eine Testgruppe zu erhalten, musst du alle Testfälle in der Testgruppe lösen.
Deine endgültige Punktzahl ist die maximale Punktzahl für eine einzelne Einsendung.

Für Gruppe~$3$ ist deine Punktzahl die niedrigste Punktzahl aller Testfälle in der Gruppe.
Die Punktzahl für jeden Testfall hängt von der Anzahl der Abfragen ab, die du verwendest;
weniger Abfragen sind besser:
Angenommen, du verwendest $q$ Abfragen. 
Wenn $q \le n+25$, dann bekommst du die vollen $80$~Punkte. 
Wenn $q > 3000$ ist, bekommst du keine Punkte.
Andernfalls bekommst du 
$118,2 - 12 \cdot \ln(q - n)$~Punkte, gerundet auf die nächste ganze Zahl. % constraint:scoringfunction
Für $n = 1500$ und $q = 3000$ bekommst du zum Beispiel $30$~Punkte.

\medskip
\begin{tabular}{lll}
Gruppe & Punkte & Beschränkungen\\\hline
$1$  &  $9$ & $n\leq 50$\\
$2$  &  $11$ & $n\leq 1000$\\
$3$  &  $0$--$80$ & $1000 < n\leq 1500$\\
\end{tabular}

\section*{Erklärung der Beispielinteraktion}

Beispielinteraktion $1$ zeigt eine mögliche Interaktion anhand des obigen Beispiels. 
Beachte, dass Annas und Tonys Stärken korrekt bestimmt sind.
(Esthers Stärke kann nie bestimmt werden.)
