\problemname{Staring Contest}

\illustration{.4}{img/Staring_contest.pdf}{}

\noindent
A staring contest is a classical battle of imperturbability in which two people stare into each other's eyes while maintaining a facial expression of assured serenity.
The goal is to maintain eye contact for longer than your opponent.
The contest ends when one participant breaks composure, typically by looking away, smiling, speaking, or giggling.

As a coach of the national staring contest you need to determine the imperturbability of each of your team's $n$ members for the upcoming world finals.
The $i$th athlete can maintain eye contact for exactly $a_i$ seconds, but these values are unknown to you in the beginning.
For instance, you could have a team of $n=3$ members:

\medskip
\begin{tabular}{lll}
  $i$ & Name & $a_i$\\\hline
  1 & Anna &  431 \\
  2 & Esther & 623 \\
  3 & Tony &  121\\
\end{tabular}

\medskip
When athletes $i$ and $j$ compete, the confrontation lasts exactly $\min(a_i, a_j)$ seconds, at which moment the weaker contestant breaks composure and both contestants start smiling and giggling within a fraction of a second.
For instance, if Anna competes against Esther, the contest lasts for $431$~seconds.
Importantly, to an outside observer the actual \emph{winner} of the confrontation (in this case, Esther) is impossible to determine, only the \emph{duration} of the contest is measurable.

Your goal is to estimate the values $a_1,\ldots, a_n$ using as few staring contests as possible.
Clearly, the strength of the strongest athlete can never be determined, so you are allowed to underestimate one of the $a_i$.

\subsection*{Interaction}

This is an interactive problem.
The interaction begins with you reading a single line containing the integer $n$.
You may then ask queries of the form ``\texttt{?} $i$ $j$'' such that $1\leq i\leq n$ and $1\leq j\leq n$ and $i\neq j$.
The response to a query is a single integer: the value $\min(a_i, a_j)$.
The interaction ends with you printing a single line consisting of \texttt{!} followed by $n$ estimates in the form of integers $b_1$, $\ldots$, $b_n$, separated by spaces.
This must be your final line of output.

Your submission is correct if $b_i=a_i$ for every contestant~$i$ except one, which you may underestimate.
To be precise, we require $b_i\leq a_i$ for all $1\leq i\leq n$ and allow $b_k \neq a_k$ for at most one~$k$.

The interactor is \emph{non-adaptive}, meaning that the $a_1,\ldots, a_n$ are determined before the interaction begins.

\subsection*{Constraints and Scoring}

The number $n$ of athletes satisfies 
$2\leq n\leq 1500$. % constraint:n
The imperturbability~$a_i$ of each athlete satisfies 
$1\leq a_i\leq 86\,400$, % constraint:skillbounds
they are all different. % constraint:allskillsdifferent
You can use 
at most $3000$~queries; % constraint:maxnumqueries
your final line of output, \emph{i.e.}, the line starting with \texttt{!}, is not counted as a query.

Your solution will be tested on a set of test groups, each worth a number of points.
Each test group contains a set of test cases.
To get the points for a test group you need to solve all test cases in the test group.
Your final score will be the maximum score of a single submission.

For group~$3$, your score is the minimum score among all test cases in the group.
The score for each test case depends on the number of queries you use;
fewer queries are better:
Suppose you use $q$ queries. 
If $q \le n+25$, then you get the full $80$~points. 
If $q > 3000$, then you get no points.
Otherwise, you get 
$118.2 - 12 \cdot \ln(q - n)$~points, rounded to the nearest integer. % constraint:scoringfunction
For instance, for $n = 1500$ and $q = 3000$, you get $30$~points.

\medskip
\begin{tabular}{lll}
Group & Points & Constraints\\\hline
$1$  &  $9$ & $n\leq 50$\\
$2$  &  $11$ & $n\leq 1000$\\
$3$  &  $0$--$80$ & $1000 < n\leq 1500$\\
\end{tabular}

\subsection*{Explanation of sample interactions}

Sample interaction $1$ shows a possible interaction using the above example. 
Note that Anna's and Tony's strengths are correctly determined.
(Esther's can never be determined.)
