\problemname{Staring Contest}

A staring contest is a classical contest of imperturbability in which two people stare into each other's eyes, maintaining a facial expression of assured serenity.
The goal is to maintain eye contact for longer than your opponent.
The game ends when one participant breaks composure, typically by looking away, smiling, speaking, or giggling.

As a coach of the national staring contest you need to determine the imperturbability of each of your team's $n$ members for the upcoming world finals.
The $i$th athlete can maintain eye contact for exactly $a_i > 0$ seconds.

For instance, you could have a team of $n=3$ members:

\medskip
\begin{tabular}{lll}
  $i$ & Name & $a_i$\\\hline
  1 & Anna &  431 \\
  2 & Esther & 623 \\
  3 & Tony &  121\\
\end{tabular}

\medskip
When athletes $i$ and $j$ compete, the confrontation lasts exactly $\min(a_i, a_j)$ seconds, at which moment the weaker contestant breaks composure and both contestants start smiling and giggling within a fraction of a second.
For instance, if Anna competes against Esther, the contest takes $431$~seconds.
Importantly, to an outside observer, the actual \emph{winner} (in this case, Esther) is often difficult to determine, only the \emph{duration} of the contest is measurable.

Your goal is to estimate $a_1,\ldots, a_n$ using as few staring contests as possible.
Clearly, the strength of the strongest athlete can never be determined, so you are allowed to make exactly one mistake by underestimating one of the $a_i$.

\subsection*{Interaction}

This is an interactive problem.
The interaction begins with you reading a single line containing the integer $n$.
You may then ask at most $3\,000$ queries of the form ``\texttt{?} $i$ $j$'' such that $1\leq i\leq n$ and $1\leq j\leq n$ and $i\neq j$.
The response is a single integer: the value $\min(a_i, a_j)$.
The interaction ends with you printing a single line consisting of \texttt{!} followed by $n$ estimates in the form of integers $b_1$ $\ldots$, $b_n$, separated by spaces.

Your submission is correct if $a_i=b_i$ for all $i$ except one and $a_i \ge b_i$ for that $i$.
To be precise, there can be at most one $k$ with $a_k\geq b_k$;
and for all $i$ with $i\neq k$ and $1\leq i\leq n$ you must have $a_i=b_i$.

The interactor is \emph{non-adaptive}, meaning that the $a_1,\ldots, a_n$ are determined before the interaction begins.

\subsection*{Constraints and Scoring}

The number $n$ of athletes satisfies 
$2\leq n\leq 1500$. % constraint:n
The imperturbability $a_i$ of each athlete satisfies 
$1\leq a_i\leq 86\,400$, % constraint:skillbounds
they are all different. % constraint:allskillsdifferent
You can use 
at most $3000$ queries. % constraint:maxnumrounds

Your final score will be the maximum score of a single submission.
There are three test groups, see table below.
Each test group contains a set of test cases. 
To get the points for a test group you need to solve all test cases in the test group.
The score in the third group depends on the number of queries you use;
fewer queries are better: Suppose you use $q$ queries. If $q \le n+25$, you get the full 80 points. If $q > 3000$, you get no points.
Otherwise, you get $118.2 - 12 \ln(q - n)$ points, rounded to the nearest integer. In particular, for $n = 1500$ and $q = 3000$, you get $30$ points.
Your score in the third group is the minimum score among all test cases.

\medskip
\begin{tabular}{lll}
Group & Points & Instance size\\
$1$  &  $9$ & $n\leq 50$\\
$2$  &  $11$ & $n\leq 1000$\\
$3$  &  $1$--$80$ & $1000 < n\leq 1500$\\
\end{tabular}

\subsection*{Explanation of sample interactions}


Sample interaction $1$ shows a possible interaction using the above example; note that Anna's and Tony's strengths are correctly determined.
(Esther's can never be determined.)
