\problemname{Konkurs gapienia się}

\illustration{.4}{img/Staring_contest.pdf}{}

\noindent
% A staring contest is a classical battle of imperturbability in which two people stare into each other's eyes while maintaining a facial expression of assured serenity.
Konkurs gapienia się to klasyczna bitwa niewzruszoności, w której dwie osoby wpatrują się w siebie nawzajem, zachowując przy tym wyraz twarzy o zapewnionym spokoju.
% The goal is to maintain eye contact for longer than your opponent.
Celem jest utrzymanie kontaktu wzrokowego dłużej niż twój przeciwnik.
% The contest ends when one participant breaks composure, typically by looking away, smiling, speaking, or giggling.
Konkurs kończy się, gdy jeden z uczestników złamie opanowanie, zazwyczaj odwracając wzrok, uśmiechając się, mówiąc lub chichocząc.

% As a coach of the national staring contest you need to determine the imperturbability of each of your team's $n$ members for the upcoming world finals.
Jako trener krajowego konkursu gapienia się musisz określić stopień niewzruszoności każdego z $n$ członków swojej drużyny na nadchodzące światowe finały.
% The $i$th athlete can maintain eye contact for exactly $a_i$ seconds, but these values are unknown to you in the beginning.
Sportowiec o numerze $i$ może utrzymać kontakt wzrokowy przez dokładnie $a_i$ sekund, ale te wartości są ci nieznane na początku.
% For instance, you could have a team of $n=3$ members:
Na przykład, możesz mieć drużynę składającą się z $n=3$ członków:

\medskip
\begin{tabular}{lll}
%   $i$ & Name & $a_i$\\\hline
  $i$ & Imię i nazwisko & $a_i$\\\hline
%   1 & Anna &  431 \\
 1 & Anna & 431 \\ % TODO: zdecydować czy inaczej nazwać przykładowych uczestników
%   2 & Esther & 623 \\
 2 & Estera & 623 \\
%   3 & Tony &  121\\
 3 & Tony & 121\\
\end{tabular}

\medskip
% When athletes $i$ and $j$ compete, the confrontation lasts exactly $\min(a_i, a_j)$ seconds, at which moment the weaker contestant breaks composure and both contestants start smiling and giggling within a fraction of a second.
Kiedy zawodnicy $i$ oraz $j$ rywalizują, konfrontacja trwa dokładnie $\min(a_i, a_j)$ sekund, w którym to momencie słabszy zawodnik łamie opanowanie i obaj zawodnicy w ciągu ułamka sekundy zaczynają się uśmiechać i chichotać.
% For instance, if Anna competes against Esther, the contest lasts for $431$~seconds.
Na przykład, jeśli Anna rywalizuje z Esther, zawody trwają przez $431$~sekund.
% Importantly, to an outside observer the actual \emph{winner} of the confrontation (in this case, Esther) is impossible to determine, only the \emph{duration} of the contest is measurable.
Co ważne, dla postronnego obserwatora faktyczny \emph{zwycięzca} konfrontacji (w tym przypadku, Ester) jest niemożliwy do określenia, mierzalny jest jedynie \emph{czas trwania} konkursu.

% Your goal is to estimate the values $a_1,\ldots, a_n$ using as few staring contests as possible.
Twoim celem jest oszacowanie wartości $a_1,\ldots, a_n$ przy użyciu jak najmniejszej liczby konkursów gapienia się.
% Clearly, the strength of the strongest athlete can never be determined, so you are allowed to underestimate one of the $a_i$.
Oczywiście, siła najsilniejszego zawodnika nigdy nie może być określona, więc wolno ci niedoszacować jedno z $a_i$.

\section*{Interakcja}

% This is an interactive problem.
To jest problem interaktywny.
% The interaction begins with you reading a single line containing the integer $n$.
Interakcja rozpoczyna się od przeczytania pojedynczego wiersza zawierającego liczbę całkowitą $n$.
% You may then ask queries of the form ``\texttt{?} $i$ $j$'' such that $1\leq i\leq n$ and $1\leq j\leq n$ and $i\neq j$.
Następnie możesz zadawać pytania w postaci ``\texttt{?} $i$ $j$'' takie, że $1\leq i\leq n$ oraz $1\leq j\leq n$ oraz $i\neq j$.
% The response to a query is a single integer: the value $\min(a_i, a_j)$.
Odpowiedzią na zapytanie jest jedna liczba całkowita: wartość $\min(a_i, a_j)$.
% The interaction ends with you printing a single line consisting of \texttt{!} followed by $n$ estimates in the form of integers $b_1$, $\ldots$, $b_n$, separated by spaces.
Interakcja kończy się wydrukowaniem pojedynczego wiersza składającego się ze znaku \texttt{!}, po którym następuje $n$ oszacowań w postaci liczb całkowitych $b_1$, $\ldots$, $b_n$, oddzielonych spacjami.
% This must be your final line of output.
To musi być twój ostatni wiersz wyjścia.

% Your submission is correct if $b_i=a_i$ for every contestant~$i$ except one, which you may underestimate.
Twoje zgłoszenie będzie uznane za poprawne, jeśli $b_i=a_i$ dla każdego zawodnika~$i$ z wyjątkiem jednego, którego możesz nie docenić.
% To be precise, we require $b_i\leq a_i$ for all $1\leq i\leq n$ and allow $b_k \neq a_k$ for at most one~$k$.
Tak dokładniej, wymagamy $b_i\leq a_i$ dla wszystkich $1\leq i\leq n$ i pozwalamy $b_k \neq a_k$ dla co najwyżej jednego~$k$.

% The interactor is \emph{non-adaptive}, meaning that the $a_1,\ldots, a_n$ are determined before the interaction begins.
Interaktor jest \emph{nieadaptacyjny}, co oznacza, że $a_1,\ldots, a_n$ są już określone przed rozpoczęciem interakcji.

\section*{Ograniczenia i punktacja}

% The number $n$ of athletes satisfies 
Liczba zawodników $n$ spełnia warunki 
$2\leq n\leq 1500$. % constraint:n
% The imperturbability~$a_i$ of each athlete satisfies 
Niewzruszoność~$a_i$ każdego zawodnika spełnia
$1\leq a_i\leq 86\,400$, % constraint:skillbounds
% they are all different. % constraint:allskillsdifferent
liczby są parami różne. % constraint:allskillsdifferent
% You can use 
Możesz użyć
% at most $3000$~queries; % constraint:maxnumqueries
co najwyżej $3000$~zapytań; % constraint:maxnumqueries
% your final line of output, \emph{i.e.}, the line starting with \texttt{!}, is not counted as a query.
ostatni wiersz twojego wyjścia, \emph{czyli} wiersz zaczynający się znakiem \texttt{!}, nie jest uznawany za zapytanie.

% Your solution will be tested on a set of test groups, each worth a number of points.
Twoje rozwiązanie zostanie przetestowane na zestawie grup testowych, z których każda warta jest pewną liczbę punktów.
% Each test group contains a set of test cases.
Każda grupa testowa zawiera zestaw przypadków testowych.
% To get the points for a test group you need to solve all test cases in the test group.
Aby uzyskać punkty za grupę testową musisz rozwiązać wszystkie przypadki testowe w tej grupie.
% Your final score will be the maximum score of a single submission.
Twój ostateczny wynik będzie maksymalnym wynikiem pojedynczego zgłoszenia.

% For group~$3$, your score is the minimum score among all test cases in the group.
Dla grupy~$3$, twój wynik to minimalny wynik wśród wszystkich przypadków testowych w grupie.
% The score for each test case depends on the number of queries you use;
Wynik dla każdego przypadku testowego zależy od liczby zapytań, których używasz;
% fewer queries are better:
mniejsza liczba zapytań jest lepsza:
% Suppose you use $q$ queries. 
Załóżmy, że używasz $q$ zapytań. 
% If $q \le n+25$, then you get the full $80$~points. 
Jeśli $q \le n+25$, to otrzymujesz pełne $80$~punkty. 
% If $q > 3000$, then you get no points.
Jeśli $q > 3000$, to nie dostaniesz żadnych punktów.
% Otherwise, you get 
W przeciwnym razie otrzymujesz 
% $118.2 - 12 \cdot \ln(q - n)$~points, rounded to the nearest integer. % constraint:scoringfunction
$118.2 - 12 \cdot \ln(q - n)$~punktów, zaokrąglone do najbliższej liczby całkowitej. % constraint:scoringfunction
% For instance, for $n = 1500$ and $q = 3000$, you get $30$~points.
Na przykład, dla $n = 1500$ oraz $q = 3000$ otrzymasz $30$~punktów.

\medskip
\begin{tabular}{lll}
% Group & Points & Constraints\\\hline
Grupa & Punkty & Ograniczenia\\\hline
$1$  &  $9$ & $n\leq 50$\\
$2$  &  $11$ & $n\leq 1000$\\
$3$  &  $0$--$80$ & $1000 < n\leq 1500$\\
\end{tabular}

\section*{Wyjaśnienie przykładowej interakcji}

% Sample interaction $1$ shows a possible interaction using the above example. 
Przykładowa interakcja $1$ pokazuje możliwą interakcję z wykorzystaniem powyższego przykładu. 
% Note that Anna's and Tony's strengths are correctly determined.
Zauważ, że siły Anny i Tony'ego są poprawnie określone.
% (Esther's can never be determined.)
(Moc Estera nigdy nie może być określona).
