\problemname{Spoksojimo varžybos}

\illustration{.4}{img/Staring_contest.pdf}{}

\noindent
Spoksojimo varžybos yra klasikinė šaltakraujiškumo dvikova, 
kurio\-je du varžovai spokso vienas į kitą stengdamiesi išlaikyti užtikrintą ramybę veide.
Varžybų tikslas -- išlaikyti akių kontaktą ilgiau nei priešininkas.
Dvikova baigiama, kai vienas iš dalyvių sudrums\-čia ramybės būseną. 
Įprastai tai nutinka dalyviui pažiūrėjus į šalį, nusišypsojus, prašnekus ar sukikenus.

Esate nacionalinių spoksojimo varžybų treneris, ir artėjančiam pasaulinių varžybų finalui
norite sužinoti kiekvieno iš jūsų $n$ komandos narių šaltakraujiškumo lygį.
$i$-tasis sportininkas gali išlaikyti akių kontaktą lygiai $a_i$ sekundžių, bet pradžioje šių reikšmių nežinote.
Pavyzdžiui, jūsų komandoje $n=3$ nariai:

\medskip
\begin{tabular}{lll}
  $i$ & Vardas & $a_i$\\\hline
  1 & Anna &  431 \\
  2 & Esther & 623 \\
  3 & Tony &  121\\
\end{tabular}

\medskip
Kai  varžosi $i$-tasis ir $j$-tasis sportininkai, jų akistata trunka lygiai $\min(a_i, a_j)$ sekundžių, 
po kurių silpnesnis varžovas sudrumsčia ramybę ir per sekundės dalį abu dalyviai ima šypsotis ir kikenti.
Pavyzdžiui, jei Anna varžytųsi su Esther, ši dvikova truktų $431$~sekundę.
Svarbu tai, kad išorinis stebėtojas negali nustatyti akistatos \emph{laimėtojo} (šiuo atveju, Esther).
Stebėtojas gali išmatuoti tik \emph{trukmę}.

Išsiaiškinkite $a_1,\ldots, a_n$ reikšmes surengdami kuo mažiau spoksojimo varžybų.
Aišku, stipriausiojo sportininko stiprumo neįmanoma nustatyti, tad galite nuvertinti (t. y. nurodyti mažesnę) vieną iš $a_i$ reikšmių.

\section*{Sąveika}

Užduotis yra interaktyvi.
Pradiniu momentu pateikiamas sveikasis skaičius $n$ (viena eilutė).
Tuomet galite išvesti užklausas formatu ,,\texttt{?} $i$ $j$``, kur $1\leq i\leq n$ ir $1\leq j\leq n$ bei $i\neq j$.
Atsakymas į užklausą yra vienas sveikasis skaičius: reikšmė $\min(a_i, a_j)$.
Sąveika baigiama kai išvedate vieną eilutę, kurioje yra simbolis \texttt{!} ir
$n$ spėjimų: tarpais atskirtų sveikųjų skaičių $b_1$, $\ldots$, $b_n$.  
Tai turi buti paskutinioji jūsų programos išvesties eilutė.

Sprendimas laikomas teisingu, jeigu $b_i=a_i$ kiekvienam dalyviui~$i$, išskyrus vienam, 
kurį galite nuvertinti.
Tai yra, reikalaujama, kad visiems $1\leq i\leq n$ galiotų $b_i\leq a_i$
ir daugiausiai vienam $k$ galiotų $b_k \neq a_k$.

Sąveikos programa \emph{nėra adaptyvi}. Tai reiškia, kad reikšmės $a_1,\ldots, a_n$ yra nustatomos prieš prasidedant sąveikai.

\section*{Ribojimai ir vertinimas}

Sportininkų skaičiui $n$ galioja
$2\leq n\leq 1500$. % constraint:n
Kiekvieno sportininko šaltakraujiškumo lygiui~$a_i$ galioja 
$1\leq a_i\leq 86\,400$, % constraint:skillbounds
visos šios reikšmės skirtingos. % constraint:allskillsdifferent
Galite atlikti daugiausiai $3000$~užklausų; % constraint:maxnumqueries
paskutinė programos išvesties eilutė, t.y., eilutė, prasidedanti simboliu \texttt{!}, nėra laikoma užklausa.

Jūsų sprendimas bus testuojamas su keliomis testų grupėmis, kurių kiekviena verta tam tikro skaičiaus taškų.
Kiekviena testų grupė sudaryta iš įvairių testų.
Testų grupės taškai skiriami tik išsprendus visus tos grupės testus.
Galutinis rezultatas lygus daugiausiai surinkusio sprendimo taškų skaičiui.

$3$-iosios testų grupės rezultatas lygus mažiausiam kurio nors šios grupės testo taškų skaičiui.
Kiekvieno testo taškai priklauso nuo to, kiek užklausų atliekate;
kuo mažiau užklausų, tuo geriau.
Tarkime, kad atliekate $q$ užklausų. 
Jei $q \le n+25$, tai gaunate visus $80$~taškų. 
Jei $q > 3000$, negaunate taškų.
Kitu atveju, gaunate 
$118.2 - 12 \cdot \ln(q - n)$~taškų, suapvalinus iki artimiausio sveikojo skaičiaus. % constraint:scoringfunction
Pavyzdžiui, jeigu $n = 1500$ ir $q = 3000$, gautumėte $30$~taškų.

\medskip

\begin{tabular}{lll}
Grupė & Taškai & Papildomi ribojimai\\\hline
$1$  &  $9$ & $n\leq 50$\\
$2$  &  $11$ & $n\leq 1000$\\
$3$  &  $0$--$80$ & $1000 < n\leq 1500$\\
\end{tabular}
\section*{Sąveikos pavyzdžio paaiškinimas}
Žemiau (\emph{Sample Interaction $1$}) parodyta galima sąveika naudojant aukščiau aprašytą pavyzdį.
Pastebėkite, kad Annos ir Tony stiprumai yra nustatyti teisingai, o Esther stiprumo neįmanoma nustatyti.
