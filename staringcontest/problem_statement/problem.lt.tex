\problemname{Spoksojimo varžybos}

\illustration{.4}{img/Staring_contest.pdf}{}

\noindent
Spoksojimo varžybos yra klasikinė šaltakraujiškumo dvikova, 
kurioje du varžovai spokso vienas į kitą stengdamiesi išlaikyti užtikrintą ramybę veide.
Varžybų tikslas -- išlaikyti akių kontaktą ilgiau nei priešininkas.
Dvikova baigiama, kai vienas iš dalyvių sudrumsčia ramybės būseną, 
įprastai tai nutinka dalyviui pažiūrėjus į šalį, nusišypsojus, prašnekus ar sukikenus.

Kadangi esate nacionalinių stebėjimo varžybų treneris, artėjantiems pasaulio finalams
norite nustatyti kiekvieno iš $n$ jūsų komandos narių šaltakraujiškumo lygį.
$i$-tasis sportininkas gali išlaikyti akių kontaktą lygiai $a_i$ sekundžių, tačiau šios reikšmės jums nėra žinomos iš pradžių.
Pavyzdžiui, jūsų komanda galėtų būti sudaryta iš $n=3$ narių:

\medskip
\begin{tabular}{lll}
  $i$ & Vardas & $a_i$\\\hline
  1 & Anna &  431 \\
  2 & Esther & 623 \\
  3 & Tony &  121\\
\end{tabular}

\medskip
Kai $i$-tasis ir $j$-tasis sportininkai varžosi, jų akistata trunka lygiai $\min(a_i, a_j)$ sekundžių, 
po kurių silpnesnis varžovas sudrumsčia ramybę ir per sekundės dalį abu dalyviai ima šypsotis ir kikenti.
Pavyzdžiui, jei Anna varžytųsi su Esther, ši dvikova truktų $431$~sekundžių.
Svarbu pastebėti, jog išoriniam stebėtojui tikrojo akistatos \emph{laimėtojo} (šiuo atveju, Esther) neįmanoma nustatyti.
Stebėtojas gali išmatuoti tik \emph{trukmę}.

Jūsų tikslas -- išsiaiškinti reikšmes $a_1,\ldots, a_n$ surengiant kuo mažiau spoksojimo varžybų.
Aišku, stipriausio sportininko stiprumo neįmanoma nustatyti, tad jums leidžiama nuvertinti vieną iš $a_i$ reikšmių.

\section*{Bendravimas}

Ši užduotis yra interaktyvi.
Bendravimo pradžioje vienoje eilutėje pateikiamas sveikasis skaičius $n$.
Tuomet jūs galite spausdinti užklausas formatu ,,\texttt{?} $i$ $j$``, kur $1\leq i\leq n$ ir $1\leq j\leq n$ bei $i\neq j$.
Atsakymas į užklausą yra vienas sveikasis skaičius: reikšmė $\min(a_i, a_j)$.
Bendravimas baigiamas jums išvedus vieną eilutę, kurioje pateiktas simbolis \texttt{!},
po kurio pateikta $n$ spėjimų tarpais atskirtais sveikaisiais skaičiais $b_1$, $\ldots$, $b_n$.  
Tai turi buti paskutinė jūsų programos išvesties eilutė.

Jūsų sprendimas laikomas teisingu, jeigu galioja $b_i=a_i$ kiekvienam dalyviui~$i$, išskyrus vienam, 
kurį galite nuvertinti.
Jei tiksliau, reikalaujama, jog visiems $1\leq i\leq n$ galiotų $b_i\leq a_i$
ir daugiausiai vienam $k$ galiotų $b_k \neq a_k$.

Bendravimo programa \emph{nėra adaptyvi}. Tai reiškia, kad reikšmės $a_1,\ldots, a_n$ yra nustatomos prieš prasidedant bendravimui.

\section*{Ribojimai ir vertinimas}

Sportininkų skaičiui $n$ galioja
$2\leq n\leq 1500$. % constraint:n
Kiekvieno sportininko šaltakraujiškumo lygiui~$a_i$ galioja 
$1\leq a_i\leq 86\,400$, % constraint:skillbounds
visos šios reikšmės skirtingos. % constraint:allskillsdifferent
Galite atlikti daugiausiai $3000$~užklausų; % constraint:maxnumqueries
paskutinė programos išvesties eilutė, t.y., eilutė, prasidedanti simboliu \texttt{!}, nėra laikoma užklausa.

Jūsų sprendimas bus testuojamas su keliomis testų grupėmis, kurių kiekviena verta tam tikro skaičiaus taškų.
Kiekviena testų grupė sudaryta iš įvairių testų.
Testų grupės taškai skiriami tik išsprendus visus testus, esančius toje grupėje.
Galutinis rezultatas lygus daugiausiai surinkusio sprendimo taškų skaičiui.

$3$-iosios testų grupės rezultatas lygus mažiausiam kurio nors šios grupės testo taškų skaičiui.
Kiekvieno testo taškai priklauso nuo to, kiek užklausų atliekate;
kuo mažiau užklausų, tuo geriau.
Tarkime, kad atliekate $q$ užklausų. 
Jei $q \le n+25$, tai gaunate visus $80$~taškų. 
Jei $q > 3000$, negaunate jokių taškų.
Kitu atveju, gaunate 
$118.2 - 12 \cdot \ln(q - n)$~taškų, suapvalinus iki artimiausio sveikojo skaičiaus. % constraint:scoringfunction
Pavyzdžiui, jeigu $n = 1500$ ir $q = 3000$, gautumėte $30$~taškų.

\medskip
\begin{tabular}{lll}
Grupė & Taškai & Papildomi ribojimai\\\hline
$1$  &  $9$ & $n\leq 50$\\
$2$  &  $11$ & $n\leq 1000$\\
$3$  &  $0$--$80$ & $1000 < n\leq 1500$\\
\end{tabular}

\section*{Bendravimo pavyzdžio paaiškinimas}

Bendravimo pavyzdyje $1$ (\emph{Sample Interaction $1$}) parodytas įmanomas bendravimas naudojant aukščiau aprašytą pavyzdį.
Pastebėkite, kad Annos ir Tonio stiprumai yra nustatyti teisingai (o Estherio stiprumo neįmanoma nustatyti).
