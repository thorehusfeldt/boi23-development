\problemname{Tõttvaatamisvõistlus}

\illustration{.4}{img/Staring_contest.pdf}{}

\noindent
Tõtt vaatamine on klassikaline vankumatuse võistlus, kus kaks inimest vaatavad täiesti tõsise näoilmega üksteise silmadesse.
Eesmärk on vastasest kauem silmsidet hoida.
Matš lõpeb, kui üks osalistest enam vastu ei pea --- vaatab eemale, naeratab, hakkab rääkima või itsitama.

Riikliku tõttvaatamisvõistluse treenerina on sinu ülesanne teha tulevaste maailmameistrivõistluste tarbeks kindlaks võistkonna iga $n$ liikme vankumatus.
$i$-s võistleja suudab silmsidet hoida täpselt $a_i$ sekundit; need väärtused on aga sulle esialgu teadmata.
Näiteks võib sul olla $n=3$ liikmest koosnev võistkond:

\medskip
\begin{tabular}{lll}
  $i$ & Nimi & $a_i$\\\hline
  1 & Anna &  431 \\
  2 & Ester & 623 \\
  3 & Tõnu &  121\\
\end{tabular}

\medskip
Kui võistlejad $i$ ja $j$ omavahel võistlevad, kestab matš täpselt $\min(a_i, a_j)$ sekundit; pärast seda ei pea nõrgem võistleja enam vastu ja mõlemad võistlevad hakkavad murdosa jooksul naeratama ja
itsitama.
Näiteks kui Anna võistleb Estri vastu, kestab võistlus $431$~sekundit.
Sealhulgas ei ole välisel vaatlejal võimalik aru saada, kes matši \emph{võitis} (antud juhul Ester):
mõõdetav on vaid matši \emph{kestus}.

Sinu eesmärk on teha võimalikult väheste matšidega kindlaks väärtused $a_1,\ldots, a_n$.
On selge, et tugevaima võistleja tugevust ei saa kuidagi kindlaks teha, seega on sul lubatud ühte arvudest $a_i$ alahinnata.

\section*{Interaktsioon}

See on interaktiivne ülesanne.
Interaktsioon algab, kui loed sisendist üheainsa täisarvu $n$.
Seejärel saad sa küsida päringuid kujul ``\texttt{?} $i$ $j$'', kus $1\leq i\leq n$, $1\leq j\leq n$ ja $i\neq j$.
Päringule saad sa vastuseks üheainsa täisarvu: väärtuse $\min(a_i, a_j)$.
Interaktsioon lõpeb, kui trükid välja rea, mis koosneb sümbolist \texttt{!} ja sellele järgnevatest $n$ tühikutega eraldatud täisarvust $b_1$, $\ldots$, $b_n$.
See peab olema viimane sinu poolt trükitud väljundirida.

Vastus loetakse korrektseks, kui $b_i=a_i$ iga võistleja~$i$ kohta peale ühe, mida võid alahinnata.
Täpsemalt nõutakse, et $b_i\leq a_i$ iga $1\leq i\leq n$ kohta, kusjuures $b_k \neq a_k$ võib kehtida ülimalt ühe~$k$ korral.

Interaktor \emph{ei ole kohanemisvõimeline} ehk arvud $a_1, \ldots, a_n$ määratakse enne interaktsiooni algust.

\section*{Piirangud ja hindamine}

Võistlejate arv $n$ rahuldab tingimust
$2\leq n\leq 1500$. % constraint:n
Iga võistleja vankumatus~$a_i$ rahuldab
$1\leq a_i\leq 86\,400$, % constraint:skillbounds
kusjuures nad on kõik omavahel erinevad. % constraint:allskillsdifferent
Sa saad kasutada
ülimalt $3000$~päringut; % constraint:maxnumqueries
kusjuures viimast väljundirida, s.t. seda, mis algab sümboliga \texttt{!}, ei loeta päringuks.

Selles ülesandes on testid jagatud gruppidesse, iga grupp on väärt mingi arvu punkte.
Iga grupi eest saavad punkte vaid need lahendused, mis läbivad kõik sellesse gruppi kuuluvad testid.
Sinu lõplik skoor on esituste maksimum.

Grupis~$3$ on sinu skoor minimaalne skoor üle kõikide grupis olevate testide.
Testi eest saadav skoor sõltub kasutatud päringute arvust;
väikesem päringute arv on parem.
Oletame, et kasutad $q$ päringut.
Kui $q \le n+25$, saad kõik $80$~punkti.
Kui $q > 3000$, siis sa punkte ei saa.
Ülejäänud juhtudel saad
$118.2 - 12 \cdot \ln(q - n)$~punkti, ümardatud lähima täisarvuni. % constraint:scoringfunction
Näiteks kui $n = 1500$ ja $q = 3000$, saad $30$~punkti.

\medskip
\begin{tabular}{lll}
Grupp & Punktid & Lisapiirangud \\\hline
$1$  &  $9$ & $n\leq 50$\\
$2$  &  $11$ & $n\leq 1000$\\
$3$  &  $0$--$80$ & $1000 < n\leq 1500$\\
\end{tabular}

\section*{Näiteinteraktsiooni selgitus}

Näiteinteraktsioonis $1$ on toodud üks võimalik interaktsioon, mis vastab
ülaltoodud näitele.
Pane tähele, et Anna ja Tõnu tugevused on leitud korrektselt.
(Estri oma ei ole võimalik selgeks teha.)
