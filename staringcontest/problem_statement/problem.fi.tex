\problemname{Tuijotuskilpailu}

\illustration{.4}{img/Staring_contest.pdf}{}

\noindent
%A staring contest is a classical battle of imperturbability in which two people 
Tuijotuskilpailu on klassinen häiriintymättömyyden taistelu, jossa kaksi ihmistä
%stare into each other's eyes while maintaining a facial expression of assured serenity.
tuijottavat toisiaan silmiin ylläpitäen samalla varman neutraalia ilmettä.
%The goal is to maintain eye contact for longer than your opponent.
Tavoitteena on ylläpitää katsekontaktia pidempään kuin vastustaja.
% The contest ends when one participant breaks composure, typically 
Kilpailu loppuu kun toinen osallistujista ei saa enää pidettyä pokkaansa,
% by looking away, smiling, speaking, or giggling.
vaan esimerkiksi katsoo pois, hymyilee, puhuu tai nauraa.

% As a coach of the national staring contest you need to determine the 
% imperturbability of each of your team's $n$ members for the upcoming world finals.
Kansallisen tuijotuskilpailun valmentajana sinun täytyy selvittää jokaisen 
tiimisi $n$:n jäsenen häiriintymättömyys tulevaa maailmanmestaruuskilpailua varten.

% The $i$th athlete can maintain eye contact for exactly $a_i$ seconds, 
Kilpailijoista $i$:s pystyy ylläpitämään katsekontaktia tasan $a_i$ sekuntia,
% but these values are unknown to you in the beginning.
mutta et tiedä näitä arvoja alussa.
% For instance, you could have a team of $n=3$ members:
Esimerkiksi, sinulla voisi olla tiimi, jossa on $n=3$ jäsentä:

\medskip
\begin{tabular}{lll}
  $i$ & Nimi & $a_i$\\\hline
  1 & Anna &  431 \\
  2 & Esther & 623 \\
  3 & Tony &  121\\
\end{tabular}

\medskip
% When athletes $i$ and $j$ compete, the confrontation lasts exactly 
%$\min(a_i, a_j)$ seconds, at which moment the weaker contestant breaks 
%composure and both contestants start smiling and giggling within a 
%fraction of a second.
Kun kisajaat $i$ ja $j$ kilpailevat, heidän yhteenottonsa kestää tarkalleen 
$\min(a_i, a_j)$ sekuntia, minkä jälkeen heikomman kisaajan pokka pettää 
ja molemmat alkavat hymyillä ja kikattaa sekunnin murto-osassa.
% For instance, if Anna competes against Esther, the contest lasts for $431$~seconds.
Jos esimerkiksi Anna kilpailee Estheriä vastaan, kisa kestää $431$~sekuntia.
% Importantly, to an outside observer the actual \emph{winner} of the confrontation 
%(in this case, Esther) is impossible to determine, only the \emph{duration} of the 
%contest is measurable.
On tärkeää huomata, että ulkopuolisen tarkastelijan on mahdotonta päätellä 
kisan \emph{voittajaa} (tässä tapauksessa Esther), vaan vain kisan 
\emph{pituus} on mitattavissa.


% Your goal is to estimate the values $a_1,\ldots, a_n$ using as few staring contests 
% as possible.
Tehtäväsi on arvioida arvot $a_1,\ldots, a_n$ käyttäen mahdollisimman pientä 
määrää tuijotuskilpailuja.
% Clearly, the strength of the strongest athlete can never be determined, so you are 
% allowed to underestimate one of the $a_i$.
Parhaimman kisaajan häiriintymättömyyttä on selvästi mahdoton määrittää, joten 
saat aliarvoida yhden arvoista $a_i$.

\section*{Interaktio}

% This is an interactive problem.
Tämä on interaktiivinen tehtävä.
% The interaction begins with you reading a single line containing the integer $n$.
Interaktio alkaa sillä, että luet yhden rivin joka sisältää kokonaisluvun $n$.
% You may then ask queries of the form ``\texttt{?} $i$ $j$'' such that 
% $1\leq i\leq n$ and $1\leq j\leq n$ and $i\neq j$.
Voit tehdä kyselyitä muodossa ``\texttt{?} $i$ $j$'', jossa 
$1\leq i\leq n$, $1\leq j\leq n$ ja $i\neq j$.
% The response to a query is a single integer: the value $\min(a_i, a_j)$.
Vastaus kyselyyn on yksi kokonaisluku: arvo $\min(a_i, a_j)$.
% The interaction ends with you printing a single line consisting of \texttt{!} 
% followed by $n$ estimates in the form of integers $b_1$, $\ldots$, $b_n$, 
% separated by spaces.
Interaktio loppuu, kun tulostat yhden rivin joka alkaa merkillä \texttt{!}, 
jota seuraa $n$ arviota välilyönneillä erotettuina kokonaislukuina 
$b_1$, $\ldots$, $b_n$.
% This must be your final line of output.
Tämän täytyy olla tulosteesi viimeinen rivi.

% Your submission is correct if $b_i=a_i$ for every contestant~$i$ except one, 
% which you may underestimate.
Palautuksesi on oikein, jos $b_i=a_i$ jokaiselle kisaajalle ~$i$ paitsi yhdelle,
jonka arvon saat aliarvoida.
%To be precise, we require $b_i\leq a_i$ for all $1\leq i\leq n$ and allow 
%$b_k \neq a_k$ for at most one~$k$.
Tarkalleen ottaen vaadimme että $b_i\leq a_i$ kaikille $1\leq i\leq n$  
ja sallimme $b_k \neq a_k$ enintään yhdelle~$k$:lle.

% The interactor is \emph{non-adaptive}, meaning that the $a_1,\ldots, a_n$ are 
% determined before the interaction begins.
Tehtävän interaktori on \emph{ei-adaptiivinen}, eli arvot $a_1,\ldots, a_n$ 
on päätetty etukäteen ennen kuin interaktio alkaa.

\section*{Rajoitukset ja pisteytys}

%The number $n$ of athletes satisfies 
%$2\leq n\leq 1500$.
Kisaajien määrälle $n$ pätee 
$2\leq n\leq 1500$. % constraint:n
%The imperturbability~$a_i$ of each athlete satisfies 
Kisaajien häiriintymättömyyksille~$a_i$ pätee
$1\leq a_i\leq 86\,400$, % constraint:skillbounds
%they are all different. 
ja ne ovat kaikki erisuuria keskenään. % constraint:allskillsdifferent
% You can use 
% at most $3000$~queries; 
% your final line of output, \emph{i.e.}, the line starting with \texttt{!}, is not counted as a query.
Voit käyttää enintään $3000$~kyselyä; % constraint:maxnumqueries
Viimeistä tulosteriviäsi, \emph{eli} sitä riviä joka alkaa merkillä \texttt{!},
ei lasketa kyselyksi.
% your final line of output, \emph{i.e.}, the line starting with \texttt{!}, is 
% not counted as a query.

% Your solution will be tested on a set of test groups, each worth a number of points.
Ratkaisu testataan testiryhmillä, joista kullakin on oma pistemäärä.
% Each test group contains a set of test cases.
Jokainen testiryhmä sisältää joukon testitapauksia.
% To get the points for a test group you need to solve all test cases in the test group.
Ryhmän pisteet saa vain, jos ratkaisee kaikki sen testitapaukset.
% Your final score will be the maximum score of a single submission.
Tehtävän lopullinen pistemäärä on suurin yksittäisen lähetyksen pistemäärä.

%For group~$3$, your score is the minimum score among all test cases in the group.
Ryhmässä~$3$ pistemääräsi on kaikkien ryhmän testien pienin pistemäärä.
%The score for each test case depends on the number of queries you use;
%fewer queries are better:
Yhden testitapauksen pistemäärä riippuu käyttämiesi kyselyiden määrästä; 
mitä vähemmän kyselyita, sitä parempi:
%Suppose you use $q$ queries. 
Oletetaan, että teet $q$ kyselyä.
%If $q \le n+25$, then you get the full $80$~points. 
Jos $q \le n+25$, niin saat täydet $80$~pistettä. 
Jos $q > 3000$, niin saat nolla pistettä.
%Otherwise, you get 
%$118.2 - 12 \cdot \ln(q - n)$~points, rounded to the nearest integer. 
Muussa tapauksessa saat $118.2 - 12 \cdot \ln(q - n)$~pistettä, 
pyöristettynä lähimpään kokonaislukuun. % constraint:scoringfunction
%For instance, for $n = 1500$ and $q = 3000$, you get $30$~points.
Jos esimerkiksi $n = 1500$ ja $q = 3000$, niin saat $30$~pistettä.

\medskip
\begin{tabular}{lll}
Ryhmä & Pisteet & Rajoitukset\\\hline
$1$  &  $9$ & $n\leq 50$\\
$2$  &  $11$ & $n\leq 1000$\\
$3$  &  $0$--$80$ & $1000 < n\leq 1500$\\
\end{tabular}

% \section*{Explanation of sample interactions}
\section*{Selitys esimerkki-interaktioille}

% Sample interaction $1$ shows a possible interaction using the above example. 
% Note that Anna's and Tony's strengths are correctly determined.
% (Esther's can never be determined.)
Esimerkki-interaktio $1$ näyttää mahdollisen interaktion ylläolevalle esimerkille.
Huomaa, että Annan ja Tonyn häiriintymättömyydet on määritetty oikein. 
(Estherin häiriintymättömyyttä ei voida määrittää.)
