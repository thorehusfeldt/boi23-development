\problemname{Staring Contest}

\illustration{.4}{img/Staring_contest.pdf}{}

\noindent
% A staring contest is a classical battle of imperturbability in which two people stare into each other's eyes while maintaining a facial expression of assured serenity.
% The goal is to maintain eye contact for longer than your opponent.
% The contest ends when one participant breaks composure, typically by looking away, smiling, speaking, or giggling.
Змагання у триманні погляду --- це класична битва за невразливість, в якій двоє людей дивляться один на одного в очі, зберігаючи обличчя впевненої безтурботності.
Мета полягає в тому, щоб тримати зоровий контакт довше, ніж опонент.
Змагання закінчується, коли один учасник втрачає самовладання, зазвичай, зводячи погляд, посміхаючись, говорячи або хихотячи.

% As a coach of the national staring contest you need to determine the imperturbability of each of your team's $n$ members for the upcoming world finals.
% The $i$th athlete can maintain eye contact for exactly $a_i$ seconds, but these values are unknown to you in the beginning.
% For instance, you could have a team of $n=3$ members:
Як тренер збірної команди з тримання погляду, вам потрібно визначити невразливість кожного з $n$ членів Вашої команди на майбутніх світових фіналах.
$i$-тий спортсмен може тримати очний контакт рівно $a_i$ секунд, але ці значення на початку Вам не відомі.
Наприклад, у Вас може бути команда з $n=3$ членів:

\medskip
\begin{tabular}{lll}
%  $i$ & Name & $a_i$\\\hline
  $i$ & Ім'я & $a_i$\\\hline
%  1 & Anna &  431 \\
%  2 & Esther & 623 \\
%  3 & Tony &  121\\
  1 & Анна &  431 \\
  2 & Естер & 623 \\
  3 & Тоні &  121\\
\end{tabular}

\medskip
% When athletes $i$ and $j$ compete, the confrontation lasts exactly $\min(a_i, a_j)$ seconds, at which moment the weaker contestant breaks composure and both contestants start smiling and giggling within a fraction of a second.
% For instance, if Anna competes against Esther, the contest lasts for $431$~seconds.
% Importantly, to an outside observer the actual \emph{winner} of the confrontation (in this case, Esther) is impossible to determine, only the \emph{duration} of the contest is measurable.
Коли спортсмени $i$ та $j$ змагаються, конфронтація триває рівно $\min(a_i, a_j)$ секунд, після чого слабший суперник втрачає самовладання, і обидва учасники відразу ж починають посміхатися та хихотіти.
Наприклад, якщо Анна змагається проти Естер, змагання триває 431 секунду.
Важливо підкреслити, що для стороннього спостерігача справжнього \emph{переможця} конфронтації (у цьому випадку --- Естер) визначити неможливо. Можливо лише виміряти \emph{тривалість} змагання.

% Your goal is to estimate the values $a_1,\ldots, a_n$ using as few staring contests as possible.
% Clearly, the strength of the strongest athlete can never be determined, so you are allowed to underestimate one of the $a_i$.
Ваша мета --- оцінити значення $a_1, \ldots, a_n$ за допомогою якомога меншої кількості змагань з тримань погляду.
Як зрозуміло, сила найсильнішого спортсмена ніколи не може бути визначена, тому Ви можете недооцінити одне зі значень $a_i$.

% \section*{Interaction}
\section*{Взаємодія}

% This is an interactive problem.
% The interaction begins with you reading a single line containing the integer $n$.
% You may then ask queries of the form ``\texttt{?} $i$ $j$'' such that $1\leq i\leq n$ and $1\leq j\leq n$ and $i\neq j$.
% The response to a query is a single integer: the value $\min(a_i, a_j)$.
% The interaction ends with you printing a single line consisting of \texttt{!} followed by $n$ estimates in the form of integers $b_1$, $\ldots$, $b_n$, separated by spaces.
% This must be your final line of output.
Це інтерактивна задача.
Взаємодія починається з того, що Ви отримуєте один рядок з цілим числом $n$.
Далі Ви можете запитувати значення, виконуючи запити у вигляді ``\texttt{?} $i$ $j$'', де $1\leq i\leq n$, $1\leq j\leq n$ та $i\neq j$.
Відповідь на запит --- це ціле число: значення $\min(a_i, a_j)$.
Взаємодія закінчується тоді, коли Ви виводите на екран рядок, що складається з \texttt{!} та $n$ оцінок у вигляді цілих чисел $b_1$, $\ldots$, $b_n$, розділених пробілами.
Це має бути вашим останнім рядком виводу.

% Your submission is correct if $b_i=a_i$ for every contestant~$i$ except one, which you may underestimate.
% To be precise, we require $b_i\leq a_i$ for all $1\leq i\leq n$ and allow $b_k \neq a_k$ for at most one~$k$.
Ваше рішення вважається правильним, якщо $b_i=a_i$ для кожного спортсмена $i$, окрім одного, якого Ви можете недооцінити.
Для точності ми вимагаємо, щоб $b_i\leq a_i$ для всіх $1\leq i\leq n$ та дозволяємо $b_k \neq a_k$ не більше ніж для одного~$k$.

% The interactor is \emph{non-adaptive}, meaning that the $a_1,\ldots, a_n$ are determined before the interaction begins.
Взаємодія відбувається з \emph{незмінним інтерактором}, що означає, що значення $a_1, \ldots, a_n$ визначаються до початку взаємодії.

% \section*{Constraints and Scoring}
\section*{Обмеження та оцінювання}

% The number $n$ of athletes satisfies
% $2\leq n\leq 1500$. % constraint:n
% The imperturbability~$a_i$ of each athlete satisfies
% $1\leq a_i\leq 86\,400$, % constraint:skillbounds
% they are all different. % constraint:allskillsdifferent
% You can use
% at most $3000$~queries; % constraint:maxnumqueries
% your final line of output, \emph{i.e.}, the line starting with \texttt{!}, is not counted as a query.
Кількість спортсменів $n$ задовольняє умову
$2\leq n\leq 1500$. % constraint:n
Навички кожного спортсмена $a_i$ задовольняють умову
$1\leq a_i\leq 86,400$, % constraint:skillbounds
вони всі різні. % constraint:allskillsdifferent
Ви можете використовувати
не більше 3000 запитів; % constraint:maxnumqueries
останній рядок виводу, тобто рядок, що починається з символу \texttt{!}, не рахується як запит.

% Your solution will be tested on a set of test groups, each worth a number of points.
% Each test group contains a set of test cases.
% To get the points for a test group you need to solve all test cases in the test group.
% Your final score will be the maximum score of a single submission.
Ваше рішення буде перевірено на наборі тестових груп, кожна з яких оцінюється в певну кількість балів.
Кожна група тестів містить набір тестових випадків.
Щоб отримати бали за групу тестів, вам потрібно вирішити всі тестові випадки у групі.
Остаточний бал складається з максимального балу, отриманий за одне подання.

% For group~$3$, your score is the minimum score among all test cases in the group.
% The score for each test case depends on the number of queries you use;
% fewer queries are better:
% Suppose you use $q$ queries.
% If $q \le n+25$, then you get the full $80$~points.
% If $q > 3000$, then you get no points.
% Otherwise, you get
% $118.2 - 12 \cdot \ln(q - n)$~points, rounded to the nearest integer. % constraint:scoringfunction
% For instance, for $n = 1500$ and $q = 3000$, you get $30$~points.
Для групи~$3$, Ваш бал буде мінімальним балом серед всіх тестових випадків у групі.
Бали за кожен тестовий випадок залежать від кількості запитів, які ви використовуєте;
менша кількість запитів оцінюється краще:
Припустимо, що ви використали $q$ запитів.
Якщо $q \le n+25$, то ви отримуєте повний бал в $80$~балів.
Якщо $q > 3000$, то ви не отримуєте жодного балу.
В іншому випадку, ви отримуєте
$118.2 - 12 \cdot \ln(q - n)$~балів, округлених до найближчого цілого числа. % constraint:scoringfunction
Наприклад, для $n = 1500$ та $q = 3000$, ви отримуєте $30$~балів.

\medskip
\begin{tabular}{lll}
% Group & Points & Constraints\\\hline
Група & Бали & Обмеження \\\hline
$1$  &  $9$ & $n\leq 50$\\
$2$  &  $11$ & $n\leq 1000$\\
$3$  &  $0$--$80$ & $1000 < n\leq 1500$\\
\end{tabular}

% \section*{Explanation of sample interactions}
\section*{Пояснення до прикладу взаємодії}

% Sample interaction $1$ shows a possible interaction using the above example.
% Note that Anna's and Tony's strengths are correctly determined.
% (Esther's can never be determined.)
Приклад взаємодії $1$ показує можливу взаємодію згідно з вищезазначеним прикладом.
Зверніть увагу, що сила Анни та Тоні визначені коректно.
(Сила Естер ніколи не може бути визначена.)
