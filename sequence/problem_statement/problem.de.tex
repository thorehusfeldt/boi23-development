\problemname{Folgen}

\noindent
Eine Folge positiver ganzer Zahlen $(x_1,\ldots,x_m)$ ist \emph{gut}, wenn $x_1 = 1$ und für jedes $1 < j \leq m$ entweder ${x_j=x_{j-1}+1}$ oder ${x_j=x_k\cdot x_l}$ für ein $k$ und $l$ mit $0< k\leq l< j$ gilt.
Zum Beispiel sind die Folgen~$(1,1)$ und $(1,2)$ beide gut, aber die Folge~$(1,3)$ ist nicht gut.
Für $n$ gegebene ganze Zahlen $w_1,\ldots,w_n$ definieren wir
das \emph{Gewicht} einer ganzzahligen Folge $(x_1,\ldots,x_m)$, die $1\leq x_j \leq n$ für jedes $1\leq j\leq m$ erfüllt, als
\[ w_{x_1} +\cdots +w_{x_m}\,.\] 
Sei zum Beispiel $w_1=10, w_2=42,w_3= 1$. Dann hat die Folge~$(1,1)$ das Gewicht $20$ und die Folge~$(1,3)$ das Gewicht $11$.
Für $1\leq v\leq n$ definiere $s_v$ als das kleinstmögliche Gewicht einer guten Folge, die den Wert $v$ enthält.

Deine Aufgabe ist es, die Werte $s_1,\ldots ,s_n$ zu bestimmen.

\section*{Eingabe}

Die erste Zeile der Eingabe besteht aus der ganzen Zahl $n$, der Anzahl der Gewichte.
Die nächsten $n$ Zeilen enthalten die ganzzahligen Gewichte $w_1, \ldots, w_n$.

\section*{Ausgabe}

Gib $n$ Zeilen aus, die der Reihe nach $s_1$, $\ldots$, $s_n$ enthalten.

\section*{Beschränkungen und Bewertung}

Es gilt immer
$1\leq n \leq 30\,000$ % constraint:n
und
$1\leq w_i \leq 10^6$ für jedes $1\leq i \leq n$.% constraint:wi

Deine Lösung wird an einer Reihe von Testgruppen getestet, von denen jede eine bestimmte Anzahl von Punkten wert ist.
Jede Testgruppe enthält eine Reihe von Testfällen.
Um die Punkte für eine Testgruppe zu erhalten, musst du alle Testfälle in der Testgruppe lösen.
Deine endgültige Punktzahl ist die maximale Punktzahl für eine einzelne Einsendung.

\medskip
\begin{tabular}{lll}
Gruppe & Punkte & Beschränkungen \\\hline
$1$   & $11$ & $n\leq 10$ \\
$2$   & $10$ & $n\leq 300$, $w_1=\cdots=w_n = 1$ \\
$3$   & $10$ & $n\leq 300$, $w_1=\cdots=w_n$ \\ % constraint:uniformweights
$4$   & $9$ & $n\leq 1400$, $w_1=\cdots=w_n = 1$ \\
$5$   & $45$ & $n\leq 5000$\\
$6$   & $15$ & \emph{Keine weiteren Beschränkungen}
\end{tabular}
