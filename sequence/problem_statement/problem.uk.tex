\problemname{Sequence}

\noindent
% A sequence of positive integers $(x_1,\ldots,x_m)$ is \emph{good} if  $x_1 = 1$ and for each $1 < j \leq m$ we have either $x_j=x_{j-1}+1$ or $x_j=x_k\cdot x_l$ for some $k$ and $l$ with $0< k\leq l< j$.
% For instance, the sequences~$(1,1)$ and $(1,2)$ are both good, but the sequence~$(1,3)$ is not good.
% For  $n$ given integers $w_1,\ldots,w_n$ define
% the \emph{weight} of an integer sequence $(x_1,\ldots,x_m)$ satisfying $1\leq x_j \leq n$ for each $1\leq j\leq m$ as
% \[ w_{x_1} +\cdots +w_{x_m}\,.\]
% For instance, given the weights $w_1=10,  w_2=42,w_3= 1$, the weight of the sequence~$(1,1)$ is $20$ and the weight of the sequence~$(1,3)$ is $11$.
% For $1\leq v\leq n$, define $s_v$ as the smallest possible weight of a good sequence containing the value $v$.
Послідовність $(x_1,\ldots,x_m)$ невід'ємних цілих чисел називається \emph{хорошою}, якщо $x_1 = 1$, а для кожного $1 < j \leq m$ виконується або $x_j=x_{j-1}+1$, або $x_j=x_k\cdot x_l$ для деяких $k$ та $l$ з $0< k\leq l< j$.
Наприклад, послідовності $(1,1)$ та $(1,2)$ є хорошими, а послідовність $(1,3)$ не є хорошою.
Для заданих $n$ цілих чисел $w_1,\ldots,w_n$ визначимо
\emph{вагу} цілої послідовності $(x_1,\ldots,x_m)$, що задовольняє $1\leq x_j \leq n$ для кожного $1\leq j\leq m$, як
\[ w_{x_1} +\cdots +w_{x_m}\,.\]
Наприклад, при $w_1=10$, $w_2=42$ та $w_3=1$, вага послідовності $(1,1)$ дорівнює $20$, а вага послідовності $(1,3)$ дорівнює $11$.
Для $1\leq v\leq n$ визначимо $s_v$ як найменшу можливу вагу хорошої послідовності, що містить значення $v$.

% Your task is to determine the values $s_1,\ldots ,s_n$.
Ваше завдання полягає у визначенні значень $s_1,\ldots ,s_n$.

% \section*{Input}
\section*{Вхідні дані}

% The first line of input consists of the integer $n$, the number of weights.
% The next $n$ lines contain the integer weights $w_1, \ldots, w_n$.
Перший рядок містить ціле число $n$ --- кількість ваг.
Наступні $n$ рядків містять цілі числа $w_1, \ldots, w_n$.

% \section*{Output}
\section*{Вихідні дані}

% Print $n$ lines containing $s_1$, $\ldots$, $s_n$ in order.
Виведіть $n$ рядків, що містять $s_1$, $\ldots$, $s_n$ відповідно.

% \section*{Constraints and Scoring}
\section*{Обмеження та оцінювання}

% We always have
Ми завжди маємо:
$1\leq n \leq 30\,000$ % constraint:n
% and
та
% $1\leq w_i \leq 10^6$ for each $1\leq i \leq n$.% constraint:wi
$1\leq w_i \leq 10^6$ для кожного $1\leq i \leq n$.% constraint:wi

% Your solution will be tested on a set of test groups, each worth a number of points.
% Each test group contains a set of test cases.
% To get the points for a test group you need to solve all test cases in the test group.
% Your final score will be the maximum score of a single submission.
Ваше рішення буде перевірено на наборі тестових груп, кожна з яких має певну кількість балів.
Кожна група містить набір тестових випадків.
Щоб отримати бали за групу тестів, потрібно пройти всі тестові випадки в цій групі.
Ваш кінцевий бал буде максимальною кількістю балів за одне відправлення.

\medskip
\begin{tabular}{lll}
% Group & Points & Constraints \\\hline
Група & Бали & Обмеження \\\hline
$1$   & $11$ & $n\leq 10$ \\
$2$   & $10$ & $n\leq 300$, $w_1=\cdots=w_n = 1$ \\
$3$   & $10$ & $n\leq 300$, $w_1=\cdots=w_n$ \\ % constraint:uniformweights
$4$   & $9$ & $n\leq 1400$, $w_1=\cdots=w_n = 1$ \\
$5$   & $45$ & $n\leq 5000$\\
% $6$   & $15$ & \emph{No additional constraints}
$6$   & $15$ & \emph{Немає додаткових обмежень}
\end{tabular}

