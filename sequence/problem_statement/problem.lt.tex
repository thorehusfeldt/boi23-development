\problemname{Seka}

\noindent
Natūraliųjų skaičių seka $(x_1,\ldots,x_m)$ vadinama \emph{gera}, jeigu  $x_1 = 1$ ir kiekvienam $1 < j \leq m$ galioja arba $x_j=x_{j-1}+1$, arba $x_j=x_k\cdot x_l$, kur $k$ ir $l$ yra tokie skaičiai, kuriems galioja $0< k\leq l< j$.
Pavyzdžiui, abidvi sekos~$(1,1)$ ir $(1,2)$ yra geros, bet seka~$(1,3)$ -- ne.
Kai duoti $n$ skaičių $w_1,\ldots,w_n$, tai sekos $(x_1,\ldots,x_m)$, kuriai galioja $1\leq x_j \leq n$ kiekvienam $1\leq j\leq m$,
\emph{svoris} yra lygus
\[ w_{x_1} +\cdots +w_{x_m}\,.\] 
Pavyzdžiui, tarkime, kad yra duoti svoriai $w_1=10,  w_2=42,w_3= 1$. Tuomet sekos~$(1,1)$ svoris yra $20$, o sekos~$(1,3)$ -- $11$.
Kiekvienam $1\leq i\leq n$ pažymėkime mažiausią geros sekos, kurioje yra $i$, svorį kaip $s_i$.

Jūsų užduotis yra rasti reikšmes $s_1,\ldots ,s_n$.

\section*{Įvestis}

Pirmoje įvesties eilutėje pateiktas sveikasis skaičius $n$ -- skaičius, kiek yra svorių.
Kitose $n$ eilučių pateikti svoriai -- sveikieji skaičiai $w_1, \ldots, w_n$.

\section*{Išvestis}

Išspausdinkite $n$ eilučių. Jose iš eilės pateikite $s_1$, $\ldots$, $s_n$.

\section*{Ribojimai ir vertinimas}

Visada galioja
$1\leq n \leq 30\,000$ % constraint:n
ir
$1\leq w_i \leq 10^6$ kiekvienam $1\leq i \leq n$.% constraint:wi

Jūsų sprendimas bus testuojamas su keliomis testų grupėmis, kurių kiekviena verta tam tikro skaičiaus taškų.
Kiekviena testų grupė sudaryta iš įvairių testų.
Testų grupės taškai skiriami tik išsprendus visus testus, esančius toje grupėje.
Galutinis rezultatas lygus daugiausiai surinkusio sprendimo taškų skaičiui.

\medskip
\begin{tabular}{lll}
Grupė & Taškai & Papildomi ribojimai \\\hline
$1$   & $11$ & $n\leq 10$ \\
$2$   & $10$ & $n\leq 300$, $w_1=\cdots=w_n = 1$ \\
$3$   & $10$ & $n\leq 300$, $w_1=\cdots=w_n$ \\ % constraint:uniformweights
$4$   & $9$ & $n\leq 1400$, $w_1=\cdots=w_n = 1$ \\
$5$   & $45$ & $n\leq 5000$\\
$6$   & $15$ & \emph{jokių papildomų ribojimų}
\end{tabular}
